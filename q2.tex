\documentclass{article}
\usepackage{amsthm}
\begin{document}
\section{}
The 2 main problems we have to find here are if the maze is connected and if it is acyclic, meaning it has no circles.
\\Our idea for a solution was to use a series of Union and Find operations on a Union Find Data Structure to solve these problems.\\
Our elements will be the $n$ x $n$ $= n^2$ cells in the maze and our disjoint sets will be the connected components of the maze which are disjoint by definition.  We will notice the following:\\
Argument 1. Because we chose the disjoint sets to be the connected components, our maze will be connected iff we have one connected component, meaning the number of sets in Union Find will be 1.\\
Argument 2. If we have a circle, we know we got it by removing a barrier between two cells at some point. These two cells which were separated by a barrier had to be in the same connected component (Otherwise it isn't a circle), meaning the maze has a circle iff we remove a barrier between two cells which are already in the same connected component, or set in Union Find.\\
Notice that RemoveBarrierr has a complexity of $O(1)$ so we can not actually use our Union and Find here, which is why we will also have a list of the barriers we need to remove.\\
Now let us explain our solution, show why it works and why the complexity is correct.\\
\underline{Init(n):} initialize an empty list, taking $O(1)$.\\
\underline{RemoveBarrier(($x_1,y_1),(x_2,y_2))$:} We insert the pair of cells ($x_1,y_1),(x_2,y_2))$ into our list which represents the barrier between them that is removed, Taking $O(1)$\\
\underline{IsAcyclic():} Firstly, we initialize the structure of our Union Find, it has $n^2$ elements so this takes $O(n^2)$. Now we do the following, for each pair of cells in the list, we find the set each cell belongs to. If they are different, then now that we've removed the barrier they are supposed to be in the same connected component which is the same set, so we call Union on these sets. If the sets are equal, this means that we just broke a barrier between two cells which are in the same connected component, meaning we have a circle so we return false. If we went over all the pairs in the list and we didn't return false, we return true since there isn't a circle (Argument 2 states that our argument is iff).\\
\underline{IsConnected():} What we do here is very similar to IsAcyclic() but it's a bit differen't. Like we did before, we initialize the structure, we go over each pair of cells in the list, we find the set each cell belongs to, if the cells are in different sets then we unite the sets. If they are in the same, we do nothing. At the end we check the number of sets in Union Find, from Argument 1 we know that the maze is connected iff the number of sets is 1, so we return true if it is 1 and false if it isn't.
We explained why the algorithm works and why the complexity of Init(n) and RemoveBarrier(($x_1,y_1),(x_2,y_2))$ are correct.\\ IsConnected() starts of by initializing which takes $O(n^2)$ and we know there are at most $4n^2$ elements in the list since we have $n^2$ cells and each cell is adjacent to 4 cells and we know we only have pairs of adjacent cells in the list. We call for Find twice and for Union once at most $4n^2$ times and we saw in the lecture that in the worst case the complexity for a series of calls of Find/Union like this is bounded by $O(4n^2log$*$(n^2))$ = $O(n^2log$*$(n^2))$. Overall, IsConnected() takes $O(n^2log$*$(n^2))$, and IsAcyclic() has the same complexity for the exact same reasons.
\end{document}
